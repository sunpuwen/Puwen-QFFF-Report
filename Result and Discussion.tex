\documentclass[12pt]{article}
\usepackage{graphicx}

\usepackage{amsmath,amssymb}
\usepackage{mathrsfs}
\usepackage[mathscr]{euscript}
\usepackage{bbold}
\usepackage{esvect}
\usepackage{amsfonts}
\usepackage{physics}
\linespread{2}


\begin{document}

\title{Result and Discussion}
\author{Puwen Sun}
\date{}
\maketitle

\section{Result and Discussion}

\begin{figure}
 \centering
\includegraphics{figure_3.png}
\caption{Log Likelihood against Mass of DM and Mass S (or A) and Mass Splitting between S and A}\label{SDDM}
\end{figure}


\begin{figure}
 \centering
\includegraphics{figure_5.png}
\caption{Log Likelihood against Mass of DM and Mass S (or A) and coupling constant $\lambda_5$ } \label{SDDMpara}
\end{figure}



DirectDM was used to take as input the Wilson coefficients describing dark matter particle interacting with quarks, and RG evolve the coupling down from electroweak scale down to nucleus scale. Heavy quarks were integrated out during the process and matched from 5-flavour basis into 4-flavour basis , and then matched from 4-flavour basis into 3-flavour basis at respective scale. 

DDCalc was then used to calculate detection rates. The match result of EFT Wilson coefficients from DirectDM were then taken as input and DDCalc provides the non-relativistic matching and the expression of the nuclear responses. Xenon1T(2018) data were then compared to the result of the expected rate calculation. 


From the discussion of inelastic dark matter, we see that parameters that are relevant, and therefore can be constraint from direct detection is the mass splitting $\delta$ and mass of dark matter $M_S$ or $M_A$. The relationship between $\delta$ and parameters in the model is
$$
\lambda_5 = \frac{2M_S \delta }{v^2}
$$
where $\lambda_5$ is one of the coupling strength of four-vertex between $\chi$ and Higgs $\chi\chi \rightarrow H^\dagger H^\dagger$. In the limit of $\lambda_5 \rightarrow 0$, there would be no mass splitting between S and A. The two degrees of freedom is just the complex $\chi_0$ field and its conjugate $\bar \chi_0$. Figure (\ref{SDDM}) and (\ref{SDDMpara}) shows the lower bound on this DM number violating processes. It shows that low-$\delta$ region has been completed ruled out in the SDDM case. 

Figure(\ref{SDDM}) shows the results from scanning over the mass splitting and the dark matter mass. Figure (\ref{SDDMpara}) shows the results from scanning over the $\lambda_5$ and $M_S$, the two parameters from the model. The red exclusion curve represents at 90 percent confidence the DD experiment rules out the model in the region with darker colours. The bright yellow region means the parameter values survive the exclusion and are subject to other constraints. 

In Fig (\ref{SDDM}), the log-likelihood plot is a non-linear mapping of Fig. (\ref{SDDMpara}) into a projection  $M_S$-$\lambda_5$ plane of the original parameter space. The white region represents the fact that no constraints about this model can be learnt from DD experiment, simply because mass splitting is too large compared to the typical momentum exchange. The inelastic scatter then become impossible. 

The degrees of freedom the direct search of DM can explore is essentially two, the DM mass $M_S$ and $\delta$, which are two projected degrees of freedom in the original parameter space. In terms of those parameters, $M_S$ is $M_\chi^2+(\lambda_3+\lambda_4)\frac{v^2}{2}$ and $\delta$ is $ \frac{\lambda_5v^2}{2(\sqrt{M_\chi^2 + (\lambda_3 + \lambda_4) \frac{v^2}{2}})}$. These are essentially the two degrees the DD experiments are exploring. These constraints shown in Fig (\ref{SDDM}) is then mapped back to original parameters in Fig (\ref{SDDMpara}). 




For FDDM, the mass splitting $\delta$ is due to a small Majorana mass, and is proportional to the induced vev acquired by the $\Delta$ 
$$
m=\sqrt 2 g_2 <\Delta^0>
$$
and 
$$
\delta=2m
$$
therefore $\delta$ is directly proportional to one of the model parameter, i.e. $  \langle \Delta^0  \rangle = \frac{\delta}{2\sqrt 2g_2}$. 

\begin{figure}
\includegraphics{figure_1.png}
\caption{Log Likelihood against Mass of DM and Mass S (or A) and Mass Splitting between S and A} \label{FDDM}
\end{figure}

\begin{figure}
\includegraphics{figure_4.png}
\caption{Log Likelihood against Mass of DM and Mass S (or A) and Vev of neutral $\Delta$ component} \label{FDDMpara}
\end{figure}




The FDDM plot Fig (\ref{FDDM}) and Fig (\ref{FDDMpara}) show that mass range of 40-60 GeV being most strongly suppressed, due to the sensitivity from the DD experiment. Fig (\ref{FDDM}) shows the scanning over the mass splitting and the dark matter mass.  Fig (\ref{FDDMpara}) shows the scanning over the $\langle \Delta_0 \rangle$ and $M_D$. The simple relationship between fermion mass splitting and vev of neutral component of scalar triplet means the coincidence of these two plots. The upper bright yellow region is the region that survives the new data from DD experiment. It can be seen that $\delta$ needs to be non-zero in the whole weak scale mass range for this model to work. The strongly restricted yellow region represents that the null result of DD experiment requires the model predicts very few events in order to work. The dark green region is where this model predicts huge number of events and therefore extremely disfavoured. 


\section{Conclusion}


The small mass splitting (i.e. small $\lambda_5$) region is completely ruled out by DD experiments at 90 percent confidence. For SDDM high-mass low-splitting region is most strongly disfavoured. For FDDM, low-splitting 50 GeV mass region is most strongly disfavoured. Based on the null results from XENON1T(2018), these regions in log-likelihood plot is tremendously suppressed. In the SDDM case, this is related to the matching between the relativistic EFT describing DM interacting with five flavours of quarks at electroweak scale, and the non-relativistic EFT describing DM with nucleons. Ultimately this is related to the 4-momentum dependence from the derivative coupling. This translated into prediction of larger coupling between DM and SM at higher DM mass. In the FDDM case, the region of suppression in the log likelihood is coincident with the sensitivity of the DD experiment. 




The results of the parameter scanning show that severe constraints were put on this model in the 2D projection of the  parameter space. Without the inelastic DM,  XENON1T(2018) results almost have ruled out the whole possible mass range of WIMP in this model. Inelastic DM were introduced to accommodate conflicting result from DD of DM and annual modulation experiment. Relic-density calculation can gives further constraints based on the dark matter density. Whether this model survives full-parameter scan under collider physics and indirect detection constraints requires further analysis. 


\end{document}
