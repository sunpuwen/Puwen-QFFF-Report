\documentclass[12pt]{article}
\usepackage{amsmath,amssymb}
\usepackage{mathrsfs}
\usepackage[mathscr]{euscript}
\usepackage{bbold}
\usepackage{esvect}
\usepackage{amsfonts}
\usepackage{physics}
\linespread{2}

\begin{document}

\title{Type II Seesaw and Inelastic Dark Matter from Scalar Triplet}
\author{Puwen Sun}
\date{}
\maketitle

\section{Inelastic Dark Matter}

The lepton number conservation in SM leads to the masses of neutrinos being exactly zero at all orders. However, oscillation experiments confirmed that neutrinos mix among themselves. A minimal explanation is incorporating a heavy triplet scalar $\Delta$ to the SM of particle physics. The lepton number violating $\Delta L2$ interaction of $\Delta$ with SM given by the following Lagrangian
$$
\mathcal L \supset M_\Delta ^2 \Delta^\dagger \Delta + \frac{1}{\sqrt{2}} (\mu_H \Delta^\dagger HH + f_{\alpha\beta}\Delta L_\alpha L_\beta+h.c.)
$$
where H and L are the SM Higgs and lepton doublets respectively. After the electroweak symmetry breaking phase transition, $\Delta$ acquires a small induced vev
$$
<\Delta> = \mu_H \frac{v^2}{\sqrt 2 M_\Delta^2}
$$
where $v$ is the SM Higgs vev 246 GeV. The vev of $\Delta$ needs to satisfy condition that for $x=\frac{<\Delta>}{v}$ the $\rho$ parameter
$$
\rho = \frac{M_W^2}{M_Z^2 \cos^2 \theta} = \frac{1+2x^2}{1+4x^2} \approx 1
$$
hence $<\Delta> $ must be smaller than $\mathcal O(1)$ GeV. The second term then gives rise to Majorana mass matrix for the three flavours of light neutrinos
$$
(M_\nu)_{\alpha\beta} = \sqrt 2 f_{\alpha\beta}<\Delta> = f_{\alpha\beta}(\frac{-\mu_H v^2}{M_\Delta^2})
$$
hence wide range of $f_{\alpha\beta}$ gives rise to neutrino masses. For $f$ on $\mathcal O(1)$, the neutrino masses corresponds to $M_\Delta$ on the order of $10^{12}$ GeV. The lepton number violation is therefore in very high scale. 


The above Lagrangian is extended by including a Inert Scalar Doublet (SDDM) $\chi = (\chi^+ \chi^0)^T$ and impose a $\mathbb Z_2$  symmetry odd on $\chi$ and even on SM fields. The potential part is 
$$
\begin{aligned}
V(\Delta,H,\chi) = &
M_\Delta^2 \Delta^\dagger \Delta + \lambda_\Delta (\Delta^\dagger \Delta)^2\\
&
 + M_H^2 H^\dagger H + \lambda_H(H^\dagger H)^2 + M_\chi^2 \chi^\dagger \chi + \lambda_\chi (\chi^\dagger\chi)^2 \\
 &+
[ \mu_H \Delta^\dagger HH + \mu_\chi \Delta^\dagger \chi\chi + h.c.]
+
\lambda_3 |H|^2 |\chi|^2 + \lambda_4|H^\dagger\chi|^2 \\
&+
\frac{\lambda_5}{2}[(H^\dagger\chi)^2 + h.c.]
\end{aligned}
$$
where we have neglected the quartic terms involving $\Delta H \chi$ because they are not relevant here due to small vev of $\Delta$. The vacuum stability of the potential requires $\lambda_\Delta, \lambda_H, \lambda_\chi >0$ and $\lambda_L = \lambda_3+\lambda_4-|\lambda_5| > -2 \sqrt{\lambda_\chi \lambda_H}$. It was also assumed that $M_\chi^2 >0$ such that $\chi$ does not develop any vev. Then DM is given by the neutral component of the doublet $\chi$. 

If replacing $\chi$ by vector like fermion doublet $\psi = (\psi_{DM},\psi_-)$ of hyper-charge $Y=-\frac{1}{2}$ and keeping same $\mathbb Z_2$ odd symmetry, the neutral component of $\psi$ is the DM candidate. The Lagrangian includes
$$
-\mathcal L \supset M_\Delta^2 \Delta ^\dagger \Delta + M_D \bar \psi \psi + \frac{1}{\sqrt 2} [ \mu_H \Delta^\dagger HH + f_{\alpha\beta}\Delta L_\alpha L_\beta + g \Delta \psi \psi + h.c.]
$$
where $M_D$ is on the order $\mathcal O (100)$ GeV is the Dirac mass of $\psi$. 

The low-energy spectrum of the SDDM theory constitutes two charged scalars $\chi^{\pm}$, two real scalars, $h$, $S$, and a pseudo scalar $A$. The masses are given by 
$$
M_{\chi^{\pm}}= M_\chi^2 + \lambda_3 \frac{v^2}{2}
$$
$$
M_h^2 = 2 \lambda_H v^2
$$
$$
M_S= M_\chi^2 + (\lambda_3 + \lambda_4 + \lambda_5) \frac{v^2}{2}
$$
$$
M_A= M_\chi^2 + (\lambda_3 + \lambda_4 - \lambda_5) \frac{v^2}{2}
$$
Depending on the sign of $\lambda_5$ either S or A constitutes the DM. Assuming $\lambda_5$ being negative, then S is the lightest scalar and the next to lightest is A. Their mass difference $\Delta M^2 = M_S^2 - M_A^2 = \lambda_5 v^2$, making the relationship between $\delta$ and parameters in the model is
$$
\lambda_5 = \frac{2M_S \delta }{v^2}
$$
where $\lambda_5$ is one of the coupling strength of four-vertex between $\chi$ and Higgs. 


Direct detection of dark matter relies on detection of nuclear recoils induced by dark matter scattering elastically or inelastically inside detection region. For inert scalar doublet dark matter (SDDM), our DM candidates are the neutral components of SU(2) doublet scalar field 
\begin{equation} \label{chi}
\chi =
\begin{pmatrix}
\chi^+ \\
\chi^0
\end{pmatrix}
\end{equation}
 obeying an ad-hoc $\mathbb{Z}_2$ odd symmetry.  As a result, the only non-zero terms relevant for direct detection, at leading order, are the t-channel Z exchange between dark matter candidates and SM particles. Since typical order of magnitude of momentum exchange $q$ is small compared to Z boson mass, the scattering process can be translated into effective vertex of contact interaction between dark matter current and SM vector and axial vector current, denoted by operator $O_{1,f}^{(6)}$ and $O_{2,f}^{(6)}$ with constant Wilson coefficient at LO, as shown previously, 
\begin{equation} \label{Fermi1}
\hat O_{1,f}^{(6)} = (\chi^{0*} i \overleftrightarrow\partial_\mu \chi^{0} ) (\bar f \gamma^\mu f)
\end{equation}
and
\begin{equation}\label{Fermi2}
\hat O_{2,f}^{(6)} = (\chi^{0*} i \overleftrightarrow\partial_\mu \chi^{0} ) (\bar f \gamma^\mu \gamma_5 f)
\end{equation}
The mediator of the above interactions is the Z boson. To show that it is possible that elastic DM scattering being suppressed so only inelastic scattering appears, we expand the field $\chi^0$ as its real and imaginary part, two real scalar fields, $S$ and $A$
\begin{equation} \label{chi0}
\chi^0= \frac{S+iA}{\sqrt 2}\\
\end{equation}
and expand out the covariant derivative $D_\mu = \partial_\mu + i(g_2 T^3 \cos \theta - g_1 Y \mathbb{1} \sin \theta ) Z_\mu + ...$, where other tems inside the covariant derivatives are ignored, $T^3$ and $Y$ are the third component of electroweak generator and hyper-charge respectively, $g_2$ is the SU(2) coupling, $g_1$ is the U(1) coupling, and $\theta$ is the Weinberg angle. Keeping only the relevant terms, we get
\begin{equation} \label{UV}
\begin{aligned}
\mathcal L_{UV} 
&
\supset (D_\mu \chi)^\dag(D^\mu \chi) \\
&\supset (\partial_\mu \chi)^\dag  i Z_\mu(g_2 T^3 \cos \theta - g_1 Y \mathbb{1} \sin \theta )\chi + h.c.\\
&\supset
(\partial_\mu \chi^0)^* (-\frac{1}{2} g_2 \cos \theta -\frac{1}{2} g_1   \sin \theta ) \chi^0 + c.c.\\
&= 
- i Z_\mu \frac{1}{2} \frac{g_2}{\cos \theta}(\partial_\mu \chi^0)^* \chi^0  + c.c. \\
&= 
 -i Z_\mu \frac{1}{2} \frac{g_2}{\cos \theta}(\partial_\mu \frac{S-iA}{\sqrt 2}) \frac{S+iA}{\sqrt 2}  + c.c. \\
 &= 
 -i Z_\mu \frac{1}{4} \frac{g_2}{\cos \theta}((\partial_\mu (S-iA) (S+iA))-  (\partial_\mu (S+iA) (S-iA)))\\
  &= 
 Z_\mu \frac{1}{2} \frac{g_2}{\cos \theta}((\partial_\mu S) A-(\partial_\mu A )S)
\end{aligned}
\end{equation}
where in the third line we used hypercharge value $1/2$ for $\chi$. 

In scalar triplet model,  if coupling $\lambda_5$ is non-zero, then the mass splitting between S and A will be $\delta \ne 0$. Then depending on the sign of $\lambda_5$, the lightest scalar particle will be either S or A, as the dark matter candidate. 
To find the Wilson coefficients, Eq.(\ref{Fermi1}) and Eq.(\ref{Fermi2}) can be matched to Eq.(\ref{UV}) in the similar procedure as $G_F$ in the 4-Fermi theory, 





In a similar fashion, inelastic FDDM in triplet model assumes a hyper-charge of $-\frac{1}{2}$ for a vector like fermion doublet
$$
\psi = 
\begin{pmatrix}
\psi_{DM}\\
\psi_-
\end{pmatrix}
$$
obeying an $\mathbb{Z}_2$ odd symmetry.  As a result, for direct detection, the relevant non-zero terms, at leading order, are the t-channel Z exchange between dark matter candidates and SM quarks. In contract with SDDM, the mass splitting $m$ is provided by an induced vev ,$< \Delta^0 >$, of the neutral component of the scalar triplet, 
$$
\Delta = 
\begin{pmatrix}
\frac{\Delta^+}{\sqrt 2}& \Delta^{++}\\
\Delta^{++} & - \frac{\Delta^0}{\sqrt 2}
\end{pmatrix}
$$
 after EW symmetry breaking. This can be seen from that in FDDM, the $\Delta \psi \psi$ coupling term is defined as
\begin{multline*}
\frac{1}{\sqrt 2} g \overline{ \psi ^c} i \tau_2 \Delta \psi = - \frac{1}{2} g [\sqrt 2(\overline {\psi^c_- } \psi_- \Delta^{++}) \\
 + (\overline{ \psi^c_-}  \psi_{DM}+ \overline {\psi_{DM}^c } \psi_- )\Delta^{+} - \sqrt 2 ( \overline{ \psi_{DM}^c} \psi_{DM} ) \Delta^{0})] 
\end{multline*}
where g is coupling strength and $\tau_2$ is $\begin{pmatrix}0 & -i \\i & 0\end{pmatrix}$. From this we see the induced vev $< \Delta^0>$ gives rise to a small Majorana mass to $\psi$, $m = \sqrt 2 g < \Delta^0 >$. 

The Dirac spinor  $\psi_{DM} $ can be written as sum of two Majorana spinors $(\psi_{DM})_L $ and $(\psi_{DM})_R $. Suppose a Dirac mass of order 100GeV, this small Majorana mass of order 100keV, the Lagrangian is written as 
\begin{equation}
\begin{aligned}
\mathcal L_{DMmass} =
& M_D [ \overline{ (\psi_{DM})_L } (\psi_{DM})_R + \overline{ (\psi_{DM})_R } (\psi_{DM})_L ] \\
&+ m [ \overline{ (\psi_{DM})_L^c } (\psi_{DM})_L +  \overline{ (\psi_{DM})_R^c } (\psi_{DM})_R ] \\ 
\end{aligned}
\end{equation}
which can be written in terms of matrix form as 
$$
\mathcal L_{DMmass} =
\begin{pmatrix}
\overline{ (\psi_{DM})_R }& \overline{ (\psi_{DM})_L }
\end{pmatrix}
\begin{pmatrix}
M_D & m\\
m & M_D 
\end{pmatrix}
\begin{pmatrix}
 (\psi_{DM})_L\\
 (\psi_{DM})_R
\end{pmatrix}
$$
The mass matrix for $\psi_{DM}$ is then
$\begin{pmatrix}
M_D & m\\
m & M_D 
\end{pmatrix}$
in the basis of $ \{ (\psi_{DM})_L , (\psi_{DM})_R\} $. Diagonalising it, the mass eigenstates are 
$$
\psi_1 = \frac{i}{\sqrt 2} ((\psi_{DM})_L -  (\psi_{DM})_R)
$$
$$
\psi_2 = \frac{1}{\sqrt 2} ((\psi_{DM})_L +  (\psi_{DM})_R)
$$
with Majorana fermion mass eigenvalues $M_{\psi_1} = M_D - m$ and  $M_{\psi_2} = M_D + m$. The mass difference between the two states $\delta = 2m$ is of the order 100 keV. After expansion of the vector current $\bar \psi \gamma_\mu \psi $, it can be shown [ 01010138 ]that in this case the inelastic scattering of DM with nucleons $(\psi_{DM})_1N \rightarrow (\psi_{DM})_2N)$  via SM Z-exchange dominates the scattering process. 

\end{document}
