\documentclass[12pt]{article}
\usepackage{amsmath,amssymb}
\usepackage{mathrsfs}
\usepackage[mathscr]{euscript}
\usepackage{bbold}
\usepackage{esvect}
\usepackage{amsfonts}


\begin{document}

\title{From Quarks to Nucleons}
\author{Puwen Sun}
\date{}
\maketitle



 \section{from quarks to nucleons}
 The general theory of dark matter (DM) scattering with detection materials can be broken down to three steps. First, a specific UV model needs to be brought down to an EFT of scattering between dark matter and quarks and gluons. Then, the matching must be done between this EFT and an non-relativistic EFT describing scattering between dark matter and nucleons. The momentum-dependent form factors enter at this step.  The last step involves the input from nuclear models because nucleus is not a simple sum of individual nucleons. The first steps really depends on different dark matter theories and the type of mediators of the scattering. In this section, the discussion starts with the matching between the two EFT, and the discussion is followed in the next section of nuclear response, which are the factors one needs to multiply when calculating the scattering amplitude of DM scattering with nucleus. 
 This section I will discuss how to translate an EFT describing interactions between dark matter with quarks into an EFT describing interactions between dark matter and protons and neutrons. 
 We will see how protons and neutrons interact with dark matter depends on what types of relativistic operators from which they are mapped, and on the corresponding different types of single-nucleon currents of the nucleon at leading order (LO) in chiral expansion.
  All DM scattering on two-nucleon currents at least 1 higher orders in momentum exchange in next-to-leading order (NLO). 
  Therefore we will see by using form factors for single-nucleon currents, matching between the two EFT can be done. 
 Traditionally, there are two distinct types of DM-SM interactions, known as the spin-dependent (SD) and the spin-independent (SI) interaction. 
 However, based on recent developments described in [ 1707.06998] , [ 1708.02678 ] and previous papers referenced inside, there are far more types of couplings of DM-Nucleon coupling than these two. 
 Generally, in dark matter direct detection, typical momentum exchange for DM scattering nucleus is much smaller than the mediator mass in a typical DM model. A full set of general non-relativistic effective interactions are therefore constructed obeying Galilean Invariance, instead of Lorentz Invariance. 
 Non-perturbative matching has been performed from an effective field theory describing dark matter interactions with quarks and gluons into effective field theory of non-relativistic dark matter interacting with non-relativistic nucleons. 
 Leading and sub-leading order in chiral counting expressions are given in [ 1707.06998 ]. 
 The general result is that more than one non-relativistic effective operators can be obtained from a single partonic operator matching. 
 In translating from a more complete UV theory into a non-relativistic effective theory, keeping only one operator does not describe the scattering in dark matter direct detection. 
 The maximum momentum exchange of DM and nucleus is $q_{max}< 200 MeV$. 
 This means one can use chiral counting.  One can organise different contributions in the nucleon EFT of each of the operators coupling DM to quarks and gluons, with an expansion parameter $q/\Gamma_{ChEFT}< 0.3$ [ 1707.06998 ] , in terms of single-nucleon form factors. 
 We will show here the leading-order (LO) results in the chiral expansion of [ 1707.06998 ] and [ 1611.00368 ] . Next-to-leading order in $(q/\Gamma_{ChEFT})^2$ is discussed in [ 1707.06998 ] and not repeated here. 
 In particular, it worth noticing that in general, the use of momentum-independent coefficients in the non-relativistic EFT for DM interactions with nucleons is not always justified. 
 More complicated numerical examples can be found in [ 1707.06998 ]. We will just look at the expressions here. 
 The starting point of this section is the effective interaction Lagrangian between DM and the SM quarks, gluons and photon, in terms of higher-dimension operators
 \begin{equation}\label{lag}
 \mathcal L_\chi = \sum_{a,d} \hat C_a^{(d)} \mathcal Q_a^{(d)}
\end{equation} 
 where d in the bracket is the dimension of the operator, a is the operator label of that dimension, and 
 $$
 \mathcal{\hat C}_a^{(d)} = \frac{\mathcal C_a^{(d)}}{\Gamma^{d-4}}
 $$
  The $C_a^{(d)}$ are dimensionless Wilson coefficients, while $\Gamma$ can be identified with the mass of the mediators between DM and the SM. 
 Index d is summed over the dimensions of the operators, $d=5,6,7$. 
 Index a is summed over all possible operators of that dimension. 
Let us start with Dirac fermion dark matter. The two dimension five operators are 
 $$
 \mathcal Q_1^{(5)} = \frac {e}{8\pi^2} (\bar \chi \sigma^{\mu \nu} \chi ) F_{\mu \nu} 
 $$
 $$
  \mathcal Q_2^{(5)} = \frac {e}{8\pi^2} (\bar \chi \sigma^{\mu \nu} i \gamma_5 \chi ) F_{\mu \nu} 
$$
where $F_{\mu\nu}$ is the electromagnetic field strength tensor and $\chi$ is the DM field, assumed to be Dirac particle.The dimension-six operators are
$$
  \mathcal Q_1^{(6)} = (\bar \chi \gamma_\mu \chi ) (\bar q \gamma^\mu q)
$$
$$
  \mathcal Q_2^{(6)} = (\bar \chi \gamma_\mu \gamma_5 \chi ) (\bar q \gamma^\mu q)
$$
$$
  \mathcal Q_3^{(6)} = (\bar \chi \gamma_\mu \chi ) (\bar q \gamma^\mu \gamma_5 q)
$$
$$
  \mathcal Q_4^{(6)} = (\bar \chi \gamma_\mu \gamma_5 \chi ) (\bar q \gamma^\mu \gamma_5  q)
$$
and a subset of the dimension-seven operators 
$$
  \mathcal Q_1^{(7)} = \frac {\alpha_s}{12\pi} (\bar \chi   \chi ) G^{a \mu \nu} G^a_{\mu \nu} 
$$
$$
  \mathcal Q_2^{(7)} = \frac {\alpha_s}{12\pi} (\bar \chi  i \gamma_5  \chi ) G^{a \mu \nu} G^a_{\mu \nu} 
$$
$$
  \mathcal Q_3^{(7)} = \frac {\alpha_s}{8\pi} (\bar \chi   \chi ) G^{a \mu \nu}  \tilde G^a_{\mu \nu} 
$$
$$
  \mathcal Q_4^{(7)} = \frac {\alpha_s}{8\pi} (\bar \chi i \gamma_5   \chi ) G^{a \mu \nu} \tilde G^a_{\mu \nu} 
$$
$$
  \mathcal Q_{5,q}^{(7)} = m_q (\bar \chi  \chi ) (\bar q  q)
  $$
 $$
   \mathcal Q_{6,q}^{(7)} = m_q (\bar \chi  i \gamma_5 \chi ) (\bar q  q)
$$
$$  
 \mathcal Q_{7,q}^{(7)} = m_q (\bar \chi  \chi ) (\bar q  i \gamma_5 q)
$$
$$
   \mathcal Q_{8,q}^{(7)}  = m_q (\bar \chi  i \gamma_5 \chi ) (\bar q   i \gamma_5 q)
$$
$$
 \mathcal Q_{9,q}^{(7)} = m_q (\bar \chi \sigma^{\mu \nu} \chi ) (\bar q \sigma_{\mu \nu} q )
$$
$$
 \mathcal Q_{10,q}^{(7)} = m_q (\bar \chi i \sigma^{\mu \nu} \gamma_5 \chi ) (\bar q \sigma_{\mu \nu} q )
$$
where $G^a_{\mu \nu} $ is the QCD field strength tensor, while $\tilde G^a_{\mu \nu} = \frac{1}{2} \epsilon_{\mu \nu \rho \sigma} G^{\rho \sigma}$ is the dual of field strength tensor, and $a = 1,...,8$ are the adjoint colour indices. $q=u,d,s$ denotes the light quarks. The remaining dimension-7 operators with derivative suppression are excluded here and are included in [ 1710.10218 ]. There are also leptonic equivalents of the operators of dimension 6 and 7, with q replaced by l in each case. 

Now, the non-perturbative matching at $\mu \approx 2 GeV$ between the EFT with three quark flavours, given by Eq. \ref{lag} , and the theory of DM interacting with non-relativistic nucleons, 
\begin{equation}\label{lag2}
\mathcal L_{NR} = \sum_{i,N} c_i^N (q^2) \mathcal O_i^N
\end{equation}
where N denotes the nucleon, p or n, and i is the operator label for non-relativistic DM-nucleon operators
The matching is done using the heavy baryon chiral perturbation theory expansion [ Jenkins and Manohar ] up to the order for which the scattering amplitudes are still parametrically dominated by single-nucleon currents. 
All the relevant Galilean-invariant operators with at most two derivatives are [ 1707.06998 ]
$$
\mathcal O_1^N = \mathbb 1_\chi \mathbb 1_N
$$
$$
\mathcal O_2^N = (v_\bot )^2 \mathbb 1_\chi \mathbb 1_N
$$
$$
\mathcal O_3^N = \mathbb 1_\chi \vec S_N \cdot ( \vec v_\bot \times \frac{i \vec q}{m_N} ) 
$$
$$
\mathcal O_4^N =  \vec S_\chi \cdot  \vec S_N
$$
$$
\mathcal O_5^N = \vec S_\chi \cdot ( \vec v_\bot \times \frac{i \vec q}{m_N} )  \mathbb 1_N
$$
$$
\mathcal O_6^N = (\vec S_\chi \cdot   \frac{i \vec q}{m_N} )(   \vec S_N \cdot   \frac{i \vec q}{m_N} )
$$
$$
\mathcal O_7^N = \mathbb 1_\chi(   \vec S_N \cdot   \vec v_\bot )
$$
$$
\mathcal O_8^N = (   \vec S_\chi \cdot   \vec v_\bot )\mathbb 1_N
$$
$$
\mathcal O_9^N =    \vec S_\chi \cdot  ( \frac{i \vec q}{m_N} \times \vec S_N )
$$
$$
\mathcal O_{10}^N =   - \mathbb 1_\chi  (  \vec S_N  \cdot \frac{i \vec q}{m_N})
$$
$$
\mathcal O_{11}^N =   - (  \vec S_\chi  \cdot \frac{i \vec q}{m_N}) \mathbb 1_N
$$
$$
\mathcal O_{12}^N =  \vec S_\chi  \cdot (  \vec S_N  \times \vec v_\bot)
$$
$$
\mathcal O_{13}^N =   -  (  \vec S_\chi  \cdot \vec v_\bot  )  (  \vec S_N  \cdot \frac{i \vec q}{m_N} )
$$
$$
\mathcal O_{14}^N =   -  (  \vec S_\chi  \cdot \frac{i \vec q}{m_N} )  (  \vec S_N  \cdot \vec v_\bot )
$$
and in addition
$$
\mathcal O_{2b}^N = (\vec S_N \cdot \vec v_\bot ) ( \vec S_\chi \cdot \vec v_\bot)
$$
where N denotes nucleons, p and n. At next-to-leading order (NLO), one more operator with three derivatives arises
$$
\mathcal O_{15}^N = - ( \vec S_\chi \cdot \frac{\vec q } {m_N } ( ( \vec S_N \times \vec v_\bot ) \cdot \frac{\vec q }{m_N} )
$$
the definition of momentum exchange is 
$$
\vec q = \vec k_2 - \vec k_1 = \vec p_1 - \vec p_2
$$
$$
\vec v_\bot = (\vec p_1 + \vec p_2 ) / (2 m_\chi) -  (\vec k_1 + \vec k_2 ) / (2 m_N )  
$$
The hadronization of the relativistic operators $\mathcal Q_{1,q}^{(6) }, ... , \mathcal Q_{10,q}^{(7)}$ at LO only leads to single-nucleon currents, and thus can be described by using form factors of single nucleon currents. [ 1611.00368 ] 
The $q^2$-dependent coefficients $c_i^N$ in Eq. \ref{lag2} are given by [ 1707.06998 ] in terms of single-nucleon form factors and UV Wilson coefficients in Eq. \ref{lag}
$$
c_1^p = - \frac {\alpha} {2 \pi m_\chi } Q_p \hat C_1^{(5)} + \sum_q ( F_1^{q/p} \hat C_{1,q}^{(6)} + F_S^{q/p} \hat C_{5,q}^{(7)} ) + F_G^P \hat C_1^{(7)} - \frac { \vec q^2}{2 m_\chi m_N } \sum_q (F_{T,0}^{q/p} - F_{T,1}^{q/p} ) \hat C_{9,q}^{(7)}
$$
$$
c_4^p = - \frac {2\alpha \mu_p } { \pi m_N }  \hat C_1^{(5)} + \sum_q ( 8 F_{T,0}^{q/p} \hat C_{9,q}^{(7)} - F_A^{q/p} \hat C_{4,q}^{(6)} ) 
$$
$$
c_5^p = - \frac {2\alpha  Q_p m_N} { \pi \vec q^2 } \hat C_1^{(5)} + 2 (  F_{T,0}^{q/p} - F_{T,1}^{q/p} )\hat C_{9,q}^{(7)} 
$$
$$
c_6^p = - \frac {2\alpha \mu_p m_N } { \pi \vec q^2 }  \hat C_1^{(5)} + \sum_q (  F_{P'}^{q/p} \hat C_{4,q}^{(6)} - \frac{m_N}{m_\chi} F_P^{q/p} \hat C_{8,q}^{(7)} ) + \frac{m_N}{m_\chi} F_{\tilde G}^p \hat C_4^{(7)}
$$
$$
c_7^p = - 2 \sum    _q F_A^{q/p} \hat C_{3,q}^{(6)} 
$$
$$
c_8^p = 2 \sum_q F_1^{q/p} \hat C_{2,q}^{(6)} 
$$
$$
c_9^p = 2 \sum_q [ ( F_1^{q/p} + F_2^{q/p}  ) \hat C_{2,q}^{(6)}  +  \frac{m_N}{m_\chi} F_A^{q/p} \hat C^{(7)}_{3,q} ]
$$
$$
c_{10}^p = F_{\tilde G}^P \hat C_3^{(7)} +  \sum_q [ F_P ^{q/p} \hat C_{7,q}^{(7)}  - 2 \frac{m_N}{m_\chi} F_{T,0}^{q/p} \hat C_{10,q}^{(7)} ]
$$
$$
c_{11}^p = \frac{2\alpha}{\pi} Q_P \frac{m_N}{\vec q ^2 } \hat C_2^{(5)} + \sum_q [ 2 (F_{T,0}^{q/p} - F_{T,1}^{q/p} ) \hat C_{10,q}^{(7)} -\frac{m_N}{m_\chi} F_S^{q/p} \hat C_{6,q}^{(7)} ] - \frac{m_N}{m_\chi} F_G^P \hat C_2^{(7)}
$$
$$
c_{12}^p = -8 \sum_q F_{T,0}^{q/p} \hat C_{10,q}^{(7)}
$$
while all other coefficients vanish at LO. The lower cases on the left hand side represents the coefficients of the non-relativistic operators. The expression for protons are given here while those for neutrons are just replacing p with n. The upper cases with hat on the right hand side represents the relativistic effective operators. The expressions for the form factors are given below. They represents the six different types of single-nucleon current corresponding to scattering of DM on a single nucleon or on a pion that attaches to the nucleon. 

The non-relativistic coefficients in Lagrangian in Eq. \ref{lag2} worth some comments. In several cases, a single operator describing DM interactions with quarks and gluons matches onto more than one non-relativistic operator, already at leading order. For example, $\mathcal O_7^N$,$\mathcal O_8^N$,$\mathcal O_9^N$,$\mathcal O_{12}^N$ cannot be turned on individually as they always come accompanied with other, regardless of the UV operator they are mapped from. On the other hand, the spin-independent operator $\mathcal O_1^N$ can arise by themselves. The same is true for spin-dependent operator $\mathcal O_4^N$. Therefore, when performing direct detection analysis, the only consistent way would be to include the whole set of non-relativistic operators altogether. 

Now, in the case of scalar dark matter, the above results have been extended. The effective interactions with SM start at dimension six 
$$
\mathcal L_\phi = \hat C_a^{(6)} Q_a^{(6)} + ... 
$$
where dots means higher-dimension operators. The dimensionless Wilson coefficients $C_a^{(d)} $ follows the same notation as above
$$
\hat C_a^{(6)} = \frac{C_a^{(6)}}{\Gamma^2}
$$
The effective dimension-six operators that couple scalar DM to quarks and gluons are 
$$
\mathcal Q_{1,q}^{(6)} = (\phi^* i \overleftrightarrow\partial_\mu \phi ) ( \bar f \gamma^\mu f ) 
$$
$$
\mathcal Q_{2,q}^{(6)} = (\phi^* i \overleftrightarrow\partial_\mu \phi ) ( \bar f \gamma^\mu \gamma_5 f ) 
$$
$$
\mathcal Q_{3,q}^{(6)} = m_f (\phi^*  \phi ) ( \bar f  f ) 
$$
$$
\mathcal Q_{4,q}^{(6)} = m_f (\phi^*  \phi ) ( \bar f i\gamma_5 f ) 
$$
$$
\mathcal Q_{5,q}^{(6)} = \frac{\alpha_s}{12 \pi} (\phi^*  \phi ) G^{a\mu\nu}G_{\mu\nu}^a
$$
$$
\mathcal Q_{6,q}^{(6)} = \frac{\alpha_s}{8 \pi} (\phi^*  \phi ) G^{a\mu\nu}\tilde G_{\mu\nu}^a
$$
$$
\mathcal Q_{7,q}^{(6)} = \frac{\alpha}{ \pi} (\phi^*  \phi ) F^{\mu\nu} F_{\mu\nu}
$$
$$
\mathcal Q_{8,q}^{(6)} = \frac{\alpha}{ \pi} (\phi^*  \phi ) F^{\mu\nu}\tilde F_{\mu\nu}
$$
where $F_{\mu\nu}$ is electromagnetic field strength tensor, and the arrow in $\phi_1 \overleftrightarrow\partial_\mu \phi_2$ means $\phi_1 \overleftrightarrow\partial_\mu \phi_2 = \phi_1 \partial_\mu \phi_2 - (\partial_\mu \phi_1) \phi_2 $. q denotes light quarks. 
The matchings are
$$
c_1^N  = \sum_q ( 2 m_\phi F_1^{q/N} \hat C_{1q}^{(6)} +F_S ^{q/N} \hat C_{3q}^{(6)} ) + F_G \hat C_5 ^{(6)}
$$
$$
c_7^N  = -4  m_\phi \sum_q  F_A^{q/N} \hat C_{2q}^{(6)} 
$$
$$
c_{10}^N  = \sum_q  F_P^{q/N} \hat C_{4q}^{(6)} +F_{\tilde G} \hat C_{3q}^{(6)}
$$
The form factors are given by 
$$
<N'|\bar q \gamma^\mu q|N> = \bar u'_N [ F_1^{q/N}(q^2) \gamma^\mu + \frac{i}{2m_N} F_2^{q/N}(q^2) \sigma^{\mu\nu} q_\nu] u_N
$$
$$
<N'|\bar q \gamma^\mu \gamma_5 q|N> = \bar u'_N [ F_A^{q/N}(q^2) \gamma^\mu \gamma_5 + \frac{i}{2m_N} F_2^{q/N}(q^2) \sigma^{\mu\nu} \gamma_5 q_\nu] u_N
$$
$$
<N'| m_q \bar q q|N> = F_s^{q/N}(q^2)\bar u'_N  u_N
$$
$$
<N'| m_q \bar q i \gamma_5 q|N> = F_P^{q/N}(q^2)\bar u'_N i \gamma_5 u_N
$$
$$
<N'| \frac{\alpha_s }{12 \pi } G^{a\mu \nu } G_{\mu\nu }^{a} |N> = F_G^{N}(q^2)\bar u'_N  u_N
$$
$$
<N'| \frac{\alpha_s }{8 \pi } G^{a\mu \nu } \tilde G_{\mu\nu }^{a} |N> = F_{\tilde G} ^{N}(q^2)\bar u'_N  i \gamma_5 u_N
$$
$$
<N'|m_q \bar q \sigma^{\mu\nu} q|N> = \bar u'_N [F_{T,0}^{q/N}(q^2) \sigma^{\mu \nu } + \frac{i}{2m_N} \gamma^{ [\mu} q^{ \nu ] } F_{T,1}^{q/N} (q^2) +
\frac{i}{m_N^2} q^{ [ \mu}k_{12}^{\nu ]} F_{T,2}^{q/N} (q^2) ] u_N
$$
where nucleon states and $u_N$ should also depend on their momenta, but we did not write them here explicitly. The form factors are only functions of $q^2$. 
Roughly speaking, the form factors describes the currents content of each operator of nucleon. 


\end{document}
