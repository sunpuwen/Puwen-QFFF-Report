\documentclass[12pt]{article}
\usepackage{amsmath,amssymb}
\usepackage{mathrsfs}
\usepackage[mathscr]{euscript}
\usepackage{bbold}
\usepackage{esvect}
\usepackage{amsfonts}
\usepackage{physics}
\linespread{2}


\begin{document}

\title{Introduction}
\author{Puwen Sun}
\date{}
\maketitle

\section{Introduction}
It is fascinating that we are still ignorant about what the major constituent of the Universe is. The standard Model of particle physics has now been so successful in predicting physical processes that is reachable by current experiments. Nevertheless almost every particle theorist agrees Beyond Standard Model is likely [Jungman]. Meanwhile, the existence of dark matter is now an implication from several different perspectives. From these perspectives it can be shown that most of the mass in the Universe is some non-luminous "dark matter" of unknown composition, which is non-baryonic and consists of some new elementary particles. 

One of the most important observational evidence is the rotation curve of spiral galaxies being flat. When one plots the velocities against the radial locations, Newton's gravitational law fails to explain the shape of the velocity curve with luminous matter only.  From the gravitational effects, one observes not enough luminous matter, and a galactic dark halo of mass 3-10 times that of luminous matter is inferred.[jungman] Similar gravitational considerations on galaxies in clusters leads to an estimation that universal density of $\Omega_m \approx $ 0.1 to 0.3 is required to account for the motion of galaxies. In other words, the gravitational laws, the observed rotation curves and velocity dispersions, and the distribution of luminous matter cannot account for each other, and hence the inclusion of unknown forms of matter that have not been discovered. 

Gravitational lensing of photons, due to General Relativity, provides another sets of evidence. Photon trajectory is curved downward the gravitational potential. This means that the dependency on mass $M$ of the deflection angle can determine the total visible and dark mass concentration, on the scale of clusters of galaxies. The image distortion of background galaxies can be used to statistically reconstruct dark matter mass. The measured mass of cluster using strong lensing, meaning structures of arcs, Einstein rings and multiple images due to the 'lenses', is much larger than typical baryon mass. Weak lensing, which uses statistically detected deformation of galaxies, also provides important dark matter measurements. 

There are also theoretical reasons for dark matter existence. The epoch of structure formation would be too short if mass density were mainly contributed by luminous matter ($\Omega \lessapprox 0.01$). Structure formation would be too quick for the observed microwave background such that fluctuations in CMB being larger than those observed. [Jungman] Gravitational potential well created by dark matter is needed for baryons to fall into after recombination for baryons "catching up" with DM.  The implied value of $\Omega$ would be $\gtrapprox$ 0.3. In addition, the power spectrum from Planck Collaboration shows the relative height of the acoustic peak can determines $\Omega_b$ and $\Omega_m$, thus $\Omega_{DM}$ . The proportionality of dark matter-ordinary matter can be implied from exact shape of power spectrum of CMB, hence CMB measurements is a strong indication of dark matter. A recent value result using CMB, SNIa and BAO is $\Omega_{DM} = 0.2589 \pm 0.0022$ from Planck Collaboration.

A related theoretical consideration is the flatness problem. If the current value of $\Omega$ being of order unity, then at Planck time, it must be $1\pm 10^{-60}$. The fine-tuning of parameter leads to inflation. 
 
 On the other hand, big-bang nucleosynthesis (BBN) suggested that the baryon density $\Omega_b \lessapprox 0.1$, 0.04 by the concordance of BBN light element abundance measurements and CMB measurements. This number is too small for the matter in the Universe. For neutrinos, N-body hot dark matter simulation almost failed to reproduce the observed structure of the Universe. If neutrino species were DM which provide dark-matter density, their mass should be on $\mathcal O(30 eV)$. However, oscillation experiment requires its mass order to be 0.1eV.  Such DM is also difficult to be making up the halos of galaxies. Non-relativistic cold (or warm) matter appeared to be current dominating candidates in the missing part of the Universe. 


The Weakly Interacting Massive Particles (WIMPs) exist in a lot of Beyond Standard Model (BSM) such as supersymmetry (SUSY) as the dark matter (DM) candidate.  The mass is much higher than neutrinos and behave as cold relics. In SUSY theories, the lightest electrically neutral super-partner neutralino of mass 10 GeV to at most a few TeV is the WIMP candidate. The exponential suppression after temperature below the mass scale, 
 Boltzmann equations are solved numerically to predict the relic abundance. Typical freeze out temperature is found to be $M_\chi/10$ for mass of WIMP around 100 GeV. Number density of WIMP at freeze out is mapped to the present day. The freeze out condition is that the annihilation rate and expansion rate are balanced. 
The required $<\sigma v>$ is then precisely the same order for a typical weak scale of mass 100 GeV and strengh $\alpha_W$.  The above is very suggestive that dark matter is just another weakly interacting particle with mass at the weak scale. 
 
 In other words, the strength of dark matter interaction in the early universe required so that they produce the right relic density measured today, coincidentally approximately equal to the strength of weak interactions, and hence the WIMP miracle. The underlying theoretical interest is the ability to use relevant particle physics models to provide a WIMP candidate to explain cosmology and astrophysical observations. Until its discovery, this fascinating subject still occupies the mind of a lot of theorist, experimentalist and phenomenologist. The null result of current ton-scale direct detection experiment still continuously rules out possible parameter ranges. In fact, some most important pieces of physics are advanced by the null result of experiments, such as the non-detection of differences in speed of light advanced theory of relativity. Since the process of advancement of science is mainly through exclusion of wrong theories and building new ones, exclusion plays as important role as verifications in finding new physical laws. 





If these DM particles in Galactic halo were in part the WIMP that account for the Galactic rotation curves, then the local density is roughly 0.3 GeV $cm^{-3}$, and the WIMPs have a Maxwell-Boltzmann distribution of velocity dispersion about 220 km $s^{-1}$ and cutoff at escape velocity of 550 km $s^{-1}$. The interaction strength of the WIMP would be such that their cosmological density today is on the order $\mathcal O(1)$. This implies the interaction scale characteristic to be of electroweak. Most SUSY models have mass of the WIMP roughly between 10 GeV and several TeV. 

The WIMPs must have nonzero coupling to SM because they must annihilate into SM during the freeze-out. They will therefore scatter from nuclei. The low-background detector detects such scatter events that leads to an energy deposit about $\mathcal O(keV)$. This is the motivation for Direct Detection DM experiments. Another category of detection searches for energetic neutrinos from Solar core that is produced by WIMP annihilation. These neutrinos with energy roughly a third of WIMP mass is so energetic that they can be distinguished from standard solar neutrinos. This way is often called the indirect detection. Large amount of theoretical works have been done about calculation of rates for both categories from realistic SUSY and other BSM models. Input from SUSY, QCD, nuclear physics, astrophysics, solar physics and detector physics enter the rate calculation at different steps. It can therefore be difficult for a given SUSY/other model to compare with various direct- and indirect- experiments.
 
 


Xenon1T(2018) [1805.12562 ]  is an experiment designed to search for WIMP using a liquid xenon time projection chamber. The 278.8 days of data collection with a fiducial mass of around 1.3 ton, excluded new regions in parameter space for the WIMP-nucleon spin-independent elastic scatter cross-section for masses above 6 GeV, with strongest constraint put at 30 GeV at $4.1 \times 10^{-47} cm^2$ at 90\% confidence level. Located on average at 3600 m water-equivalent depth, it contains 3.2 t of ultra-pure liquid Xenon with 2 ton employed as target material. 127 and 121 photomultiplier tubes were instrumented above and below the cylinder to detect scintillation signal, either from a particle incident on Xe target, or from ionised electrons which were drifted and scintillated via electroluminescence. Distinguishing these two signals identifies nuclear recoils from electronic recoils, while signal delays between the two identifies the spatial location of the interaction. No significant excess above background was found, and exclusion curve for WIMP-nucleon SI cross section were extended to lower limit than PandaX(2017) II, LUX(2017) and Xenon1T(2017). Future detector upgrade, XENONnT, will increase the sensitivity by an oder of magnitude with target mass of 5.9 ton. 

Xenon1T falls into the large category of direct detection experimental efforts, in which one seeks to measure the energy deposit when a WIMP interacts with a nucleus in a detector [1101.4318 ]. DM particles are expected to pass through Earth continuously [ 1705.07920 ] with a DM halo modelled around the solar system. Using the fact that weakly interacting particles occasionally scatter with regular matter, DD experiments actively look for nuclear recoils in large volume of inert target materials places in ultra-clean environment deep underground [1705.07920], such as XENON1T. 

The differential rate of recoil event in DD experiment is 
$$
\frac{dR}{dE} = \frac{2\rho_0}{m_\chi}\int vf(v,t)\frac{d\sigma}{dq^2}(q^2,v)d^3v
$$
where $m_\chi$ is the WIMP mass, $\rho_0$ is the local DM mass density, $f(\vec v,t)$ is the time-dependent WIMP velocity distribution, $\frac{d\sigma}{dq^2} (q^2,v) $ is the velocity-dependent differential cross-section, and $q^2= 2m_{nuc}E$ is the momentum exchanged in the scattering process for a recoil energy $E$ [1705.07920]. The lower limit of the integration should be $v_{min}$ needed for a recoil inside the detector, $v_{min} = \sqrt{\frac{1}{2M_NE_R}}(\frac{M_NE_R}{\mu_n}+\delta)$ where $\mu_n$ is the WIMP-nucleus reduced mass. Most direct detection experiment contain more than one isotope. The differential rate then is given by the mass weighted sum over isotopes of the above equation. The expected number of signal events is given by 
$$
 N_p = MT\int_0^\infty \phi(E)\frac{dR}{dE}(E)dE
 $$
 where $M$ is the detector mass, $T$ is the exposure time. The detector response function $\phi(E)$ describes the the experimental consideration and always is plotted as the efficiency curve, for example, in simplest case, would be given by a lower and an upper bound on the reconstructed energy. More elaborate analyses of experiments implemented can all be captured by detector response $\phi(E)$.  Because this response of detector is always independent of the nature of particle interaction of the WIMP scattering with nuclei, the differential event rate captures all particle physics and nuclear physics. 



In this work, I will describe the scalar triplet model and the inelastic dark matter candidates raised by the model. I will show the necessary steps in the event rate calculations which translate original parameters into Wilson coefficients and then into experimental predictions.  I will also perform an inelastic DM analysis based on the scalar triplet model with dark matter mass splitting as additional parameter. Exclusion plots are shown in the result section to show the constraints imposed by XENON1T(2018) experiment data. However, the effective field theory (EFT) discussion will be general and model independent, so a general UV theory of dark matter can easily be translated into observables. I will review why and how the traditional Spin independent (SI) and spin dependent (SD) analysis did not capture all the possibilities of how dark matter interact with SM particles in the context of direct detection experiment, based on recent developments in EFT. New theoretical tools have made the calculation tractable while keeping all assumptions consistent with each other in different steps. I will demonstrate how these EFT developments help future experiments to comparing the results and constraining new theories. 



\end{document}
