\documentclass[12pt]{article}
\usepackage{amsmath,amssymb}
\usepackage{mathrsfs}
\usepackage[mathscr]{euscript}
\usepackage{bbold}
\usepackage{esvect}
\usepackage{amsfonts}
\usepackage{physics}
\linespread{2}


\begin{document}

\title{From Nucleons to Nucleus}
\author{Puwen Sun}
\date{}
\maketitle


\section{From Nucleon to Nucleus}

In the last section, we extended the set of the existing operators describing dark matter direct detection. The basic building blocks of EFT of dark matter direct detection were listed, according to Galilean-invariance and up to quadratic order in momentum transfer. We saw that any DM particle theory can be translated into a set of Wilson coefficients of an effective operators of DM and SM nucleons. However, the nucleus is not a simple sum of individual protons and neutrons. One can use the NR EFT from last section, in theory, to find detection rates if we were using free protons. The momentum exchanged during the process means modelling nucleus as point local operators is not valid in dark matter direct detection, simply due to the size of inverse nuclear size being comparable to the momentum transfer.  In this section, one more important ingredients in the calculation of differential event rates of DM detection will be discussed, the nuclear response. The current canonical way of treating nucleus as point particle with charge and spin significantly restrict possible interactions, but without justification [1308,6288 ]. The traditional form factors attempt to characterise this effect by multiplying matrix element with phenomenological forms not calculated from microscopical arguments.  Based on recent advancements in [ 1203.3542 ] and [ 1308.6288 ] and the references therein, not only form factors, but also new operators arise once the momentum transfer is large compared with inverse nuclear size. These responses differ significantly from traditional minimal WIMP cases in their coupling strengths. These recent developments will change how the results from different experiments compared against each other in the future. 

All possible DM-nucleus response functions, including those which are momentum and velocity dependent, are provided in [ 1203.3542 ]. There are six response allowed by basic symmetry considerations rather than the standard two (SI and SD). They provides important intermediate results for experimentalists to characterise DM, and to use these functions in the underlying effective theory operators that describe DM-ordinary matter interaction. Another equally important motivation of a generalised nuclear response theory and non-relativistic effective theory is that the possibility of momentum-dependent interactions being the dominant ones. This happens in the case that the leading momentum-independent interactions are suppressed by various scenarios [ 1007.5325 ], [ hep-ph/0003010 ], [ hep-ph/9310290 ]. 

As an example, non-relativistic $\mathcal O_7$ is $\vec S_N \cdot \vec v^\bot $ can be showed how to  separate $\vec v^\bot $ into the centre-of-mass component $\vec v^\bot_T $ and the other component $\vec v^\bot_N $ and for constructing the nuclear operators. 

In the limit of point nucleus, interaction being only characterised by macroscopic quantum numbers charge, spin and isospin, the operator becomes simply $\vec S_N \cdot \vec v^\bot_T $ which is 
$$
\vec S_N \cdot \vec v^\bot_T  =  \vec v^\bot_T \cdot \frac{1}{2} \sum_{j=1}^A \vec \sigma(j) 
$$
The centre-of-mass nuclear velocities $\vec v^\bot_{T,in} $ and $\vec v^\bot_{T,out} $ are just the averaged velocities of the nucleons. 

The underlying model-building assumption is that charge and current operators are one-body and local. For this operator, the sum over individual axial charge operators. 
$$
\sum_{j=1}^A \vec \sigma(j) \cdot \frac{\vec p(j) }{2m_N} =  \vec v^\bot_T \cdot \frac{1}{2} \sum_{j=1}^A \vec \sigma(j)  + [   \sum_{j=1}^A \vec \sigma(j) \cdot \frac{\vec p(j) }{2m_N} ]_{intrinsic}
$$
where the second term the intrinsic operator, characterising the internal motions of individual nucleons that is independent of the assembled motion of all nucleons. The intrinsic operator vanish for even multipoles by parity and for odd multipoles by time reversal. Consider a V-A four-fermion contact operator between dark matter and a nucleus, $(\bar \psi_\chi \gamma_\mu \psi_\chi ) (\bar \psi_N \gamma^\mu \gamma_5 \psi_N)$. 


The result of factorisation of particle physics and nuclear physics can be written
$$
\frac{1}{2j_\chi +1} \frac{1}{2j_N +1} \sum_{spins} \mid \mathcal M \mid ^2  = \sum_k \sum_{\tau=0,1}
 \sum_{\tau'=0,1} R_k ( \vec v_T^{\bot2} , \frac{\vec q^2 }{m_N^2}, \{ c_i^\tau c_j^{\tau'} \}) W_k^{\tau \tau' } (\vec q^2 b^2) 
 $$
where the sum runs over $k$,$\tau$,$\tau'$, which are labels for nuclear response types, isospin labels respectively.
The WIMP response function $R_k$ represents the DM particle physics and it encapsulates all the non-relativistic effective interactions from previous section. The response function is built from the whole set of non-relativistic operators. They depend on 2N coefficients, where N is the number of types of NR operators in the EFT. Here, the $\tau$ labels the isospin rather than n and p of the non-relativistic EFT. The convention here is that $c^{\tau=0}_i = c_i^{(n)}+ c_i^{(n)}$ and $c^{\tau = 1}_i =c_i^{(p)}-c_i^{(n)}  $. The WIMP response functions depend on the relative target velocity, $\vec v_T^\bot$, the three-momentum transfer, $\vec q = \vec p' - \vec p = \vec k - \vec k'$, where $\vec p$ and $\vec p'$ are the WIMP 3-momentum and $\vec k$ and $\vec k'$ are the nucleon 3-momentum. 

The nuclear response functions, $W_k$, on the other hand, represents the nuclear physics and it encapsulates the nuclear matrix elements of different types of response of the nucleus. While as many UV theories as there might be, at low energies, classification in terms of non-relativistic interactions do not sacrifice generality. By writing operators in terms of observable quantities, the relationship between operators and the underlying physics becomes apparent. At low energies, one can be guaranteed that effective interactions are general. Embedding these interactions in the nucleus then becomes possible as the most general nuclear response can be written down obeying an assumed set of symmetries. It will become increasing clear for a lot of questions that have not been addressed in literature, such as what is the constraints on dark matter theories and interactions can be learnt from dark matter direct detection experiments. How redundant our low-energy EFT is in the light of scattering experiments. 


\subsection{An Outline of Derivation}
The starting point of this section is writing the NR EFT in isospin 0 and isospin 1 basis, rather than n and p. 
$$
\sum_{i=1}^{15}(c_i^0 \mathbb 1 + c_i^1 \tau_3) \mathcal O_i = \sum_{\tau=0}^1\sum_{i=1}^{15}c_i^\tau \mathcal O_i t^\tau
$$
where the isospin state conventions are 
$$
\ket p = \begin{pmatrix} 1\\ 0 \end{pmatrix}
$$




The starting point of the general theory of nuclear response will be to write down the EFT Lagrangian 
$$
\mathcal L_{EFT} = c_1 \mathbb 1 + c_2 \vec v^\bot \cdot \vec v^\bot + c_3 \vec S_N \cdot (\vec q \times \vec v^\bot) + ...
$$
The previous velocity $v^\bot$ now can be separated into a centre-of-mass piece $\vec v_T ^\bot $ and the piece describing the internal relative velocities of target nucleons $\vec v_N^\bot$ (associated with the Jacobi momenta). These two velocities separately interact with spins, so the interactions are separately invariant. The assumption that nuclear ground state are good parity and CP can lead to the conclusion that there are six different types of nuclear responses. In order to determine the explicit forms of these multipole operators, further assumptions have to be made that the spins and momenta being nucleonic local operators. This is the typical one-body assumption for common nuclear physics calculations. [1203.3542] 

The $v_T^\bot$ component is associated with the point limit for a nucleus, which is fully characterised by macroscopic quantum numbers of charge, spin, and isospin. For example,  the "T" component of the $O_7$ operator $\vec v ^\bot \cdot S_N$, under this point limit,  
$$
\vec v_T^\bot \cdot \vec S_N = \vec v_T^\bot \cdot \frac{1}{2} \sum_{j=1}^A \vec \sigma(j) 
$$
where $ v_T^\bot = \frac{1}{2}(\vec v_{\chi,in} +\vec v_{\chi,out}-\vec v_{T,in}-\vec v_{T,out})$. By averaging over the velocities of individual nucleons we can get the $\vec v_{T,in}$ and $\vec v_{T,out}$. For the purpose of embedding of the $O_7$ operator, the one-body and local assumption can separate the operator into its centre-of-mass contribution, and its intrinsic (internal-relative-velocity) contribution. The former is the point limit, while the latter is the contribution associated with the $A-1$ relative Jacobi three-momenta,  inside $\vec v_N^\bot$.  The matrix elements of the intrinsic operator vanish for even multipoles by parity and for odd multipoles by time reversal for elastic scattering. In this way, the EFT easily identifies all the Galilean invariant quantities, instead of conventional method of starting with a covariant interaction. For example, starting with $\bar \psi_\chi \gamma_\mu \psi_\chi \bar \psi_N \gamma^\mu \gamma^5 \psi_N$, one can summing over charge and three-current contributions to scattering and obtain the same invariants, instead of staring with the EFT and identify the target contribution immediately. 

Arranging various terms of EFT by grouping the same types of nuclear response together, one obtains the following form 
$$
\mathcal L_{ET} = l_0 + l_0^A [ -2 \vec v_N^\bot \cdot \vec S_N] + \vec l_5 \cdot [2\vec S_N] + \vec l_M \cdot [-\vec v_N^\bot] + \vec l_E \cdot [2i \vec v_N \times \vec S_N ]
$$
$$
= l_0 1 + l_0^A \frac{\vec p_i + \vec p_f}{2m_N} \cdot \sigma + \vec l_5 \cdot \vec \sigma + \vec l_M \cdot  \frac{\vec p_i + \vec p_f}{2m_N} + \vec l_E \cdot (-i  \frac{\vec p_i + \vec p_f}{2m_N}\times \vec \sigma)
$$
where the coefficients $l_0$ (charge), $l_0^A$(axial charge), $\vec l_5$ (axial vector density), $\vec l_M$ (vector magnetic density), and $\vec l_E$(vector electric density) are given below, and they encapsulate the WIMP response from the EFT. 
This is just an elementary step of re-arrangements of terms, where the NR operators are re-written in terms of six WIMP tensors, which are re-groupings of all the operators. The six WIMP tensors that encodes all the EFT information are 
$$
l_0^\tau = c_1^\tau + i ( \frac{\vec q}{m_N} \times \vec v_T^\bot) \cdot \vec S_\chi c_5^\tau + \vec v_T^\bot \cdot \vec S_\chi c_8^\tau + i \frac{\vec q}{m_N}\cdot \vec S_\chi c_{11}^\tau
$$
$$
l_0^{A\tau} = -\frac{1}{2} [ c_7^\tau + i \frac {\vec q}{m_N} \cdot \vec S_\chi c_{14}^\tau ]
$$
$$
\vec l_5 = \frac{1}{2} [ i \frac{\vec q}{m_N} \times \vec v_T^\bot c_3^\tau + \vec S_\chi c_4^\tau + 
\frac{\vec q}{m_N}\frac{\vec q}{m_N}\cdot \vec S_\chi c_6 ^\tau + \vec v _T ^\tau c_7 ^\tau + i \frac{\vec q}{m_N} \times \vec S_\chi c_9^\tau 
+ i \frac{\vec q }{m_N} c_{10}^\tau 
+ \vec v_T^\bot \times \vec S_\chi c_{12}^\tau 
$$
$$+ i \frac{\vec q }{m_N} \vec v_T^\bot \cdot \vec S_\chi c_{13}^\tau + i \vec v_T^\bot \frac{\vec q }{m_N} \cdot \vec S_\chi c_{14}^\tau + \frac{\vec q }{m_N} \times \vec v_T^\bot \frac{\vec q}{m_N} \cdot S_\chi c_{15}^\tau]
$$
$$
\vec l_M = i \frac{\vec q}{m_N} \times \vec S_\chi c_5^\tau - \vec S_\chi c_8^\tau
$$
$$
\vec l_E = \frac{1}{2} [ \frac{\vec q}{m_N} c_3^\tau + i \vec S_\chi c_{12}^\tau - \frac{\vec q}{m_N}\times \vec S_\chi c_{13}^\tau - i \frac{\vec q}{m_N}\frac{\vec q}{m_N} \cdot \vec S_\chi c_{15}^\tau]
$$
By substitutions of the relevant quantities with operators, one gets Hamiltonian density, where one also sums every nucleonic operator in the nucleus. The Hamiltonian density then decided the six types of nuclear responses multipole operators $M_J$ (vector charge), 
$\Sigma_J''$ (axial longitudinal electric), 
$\Sigma_J'$ (axial transverse electric) ,
 $\Delta_{JM}$ (vector transverse) ,
  $\tilde \Phi'_{JM;\tau}$ (vector transverse electric),
   $ \Phi_{JM}^{\tau''} $ (vector longitudinal electric) ,
    all of which are functions of $q\vec x$, and $M_{JM}$ and $\vec M_{JL}^M $ are the spherical harmonics and vector spherical harmonics respectively. 

Now we are ready to give the most important formula for this section. The following gives the cross section as a sum of products of WIMP $R_k$ and nuclear $W_k$ response functions. 
$$
\begin{aligned}
\frac{1}{2j_\chi + 1}\frac{1}{2j_\chi + 1} \sum_{spins} | \mathcal M | ^2 = 
&
\frac{4\pi }{2j_N +1}\sum_{\tau=0,1}\sum_{\tau=0,1}
\{
[R^{\tau\tau'}_M(\vec v_T^{\bot2},\frac{\vec q^2}{m_N^2})
 W_M(y)
 \\
 &
+ R^{\tau\tau'}_{\Sigma''} (\vec v_T^{\bot2},\frac{\vec q^2}{m_N^2})
W_{\Sigma''}(y)
+ R^{\tau\tau'}_{\Sigma'} (\vec v_T^{\bot2},\frac{\vec q^2}{m_N^2})
W_{\Sigma'}(y)
]\\
&
+\frac{\vec q}{m_N^2}[
R^{\tau\tau'}_{\Phi''} (\vec v_T^{\bot2},\frac{\vec q^2}{m_N^2})
W_{\Sigma''}(y)
+ R^{\tau\tau'}_{\Phi''M} (\vec v_T^{\bot2},\frac{\vec q^2}{m_N^2})
W_{\Phi''M}(y)
 \\
 &
+ R^{\tau\tau'}_{\Phi'} (\vec v_T^{\bot2},\frac{\vec q^2}{m_N^2})
W_{\Phi'}(y)
+ R^{\tau\tau'}_{\Delta} (\vec v_T^{\bot2},\frac{\vec q^2}{m_N^2})
W_{\Delta}(y)
 \\
 &
+ R^{\tau\tau'}_{\Delta\Sigma'} (\vec v_T^{\bot2},\frac{\vec q^2}{m_N^2})
W_{\Delta\Sigma'}(y)
]
\}
\end{aligned}
$$
where the WIMP response function is given by matching terms after square and given here
$$
R_M ^{\tau \tau'} ( \vec v_T^{\bot2}
, \frac{\vec q^2}{m_N^2})
= c_1^\tau c_1^{\tau' } + \frac{j_\chi (j_\chi +1)}{3} [\frac{\vec q^2}{m_N^2} 
\vec v_T^{\bot2} c_5^\tau c_5^{\tau'}
+\vec v_T^{\bot2} c_8^\tau c_8^{\tau'}+\frac{\vec q}{m_N^2} \vec v_T^{\bot2} c_{11}^\tau c_{11}^{\tau'} ]
$$
$$
R_{\Phi''} ^{\tau \tau'} ( \vec v_T^{\bot2}, \frac{\vec q^2}{m_N^2})
= \frac{\vec q^2}{4m_N^2} c_3^\tau c_3^{\tau' } + \frac{j_\chi (j_\chi +1)}{12} (
c_{12}^\tau - \frac{\vec q}{m_N^2}c_{15}^\tau)(c_{12}^{\tau'} - \frac{ \vec q^2 }{ m_N^2 } c_{15}^{\tau'})
$$
$$
R_{\Phi''M} ^{\tau \tau'} ( \vec v_T^{\bot2}, \frac{\vec q^2}{m_N^2})
=  c_3^\tau c_1^{\tau' } + \frac{j_\chi (j_\chi +1)}{3} (
c_{12}^\tau - \frac{\vec q}{m_N^2}c_{15}^\tau)(c_{11}^{\tau'})
$$
$$
R_{\tilde \Phi'} ^{\tau \tau'} ( \vec v_T^{\bot2}, \frac{\vec q^2}{m_N^2})
=  \frac{j_\chi (j_\chi +1)}{12} (
c_{12}^\tau c_{12}^{\tau'} - \frac{\vec q}{m_N^2}c_{13}^\tau)(c_{13}^{\tau'})
$$
$$
R_{ \Sigma''} ^{\tau \tau'} ( \vec v_T^{\bot2}, \frac{\vec q^2}{m_N^2})
=  \frac{\vec q^2}{4m_N^2} c_{10}^\tau c_{10}^{\tau' } +\frac{j_\chi (j_\chi +1)}{12} [
c_{4}^\tau c_{4}^{\tau'} + \frac{\vec q^2}{m_N^2}(c_{4}^\tau c_{6}^{\tau'}+c_6^\tau c_{4}^{\tau'})+\frac{\vec q^4}{m_N^4}c_{6}^\tau c_{6}^{\tau'}
$$
$$
+ v_T^{\bot2}c_{12}^\tau c_{12}^{\tau'}
\frac{\vec q^2}{m_N^2}c_{13}^\tau c_{13}^{\tau'}
]
$$
$$
R_{ \Sigma'} ^{\tau \tau'} ( \vec v_T^{\bot2}, \frac{\vec q^2}{m_N^2})
= \frac{1}{8}[ \frac{\vec q^2}{m_N^2} \vec v_T^{\bot2} c_{3}^\tau c_{3}^{\tau' } +\vec v_T^{\bot2} c_{7}^\tau c_{7}^{\tau' } ]
+ \frac{j_\chi (j_\chi +1)}{12} [
c_{4}^\tau c_{4}^{\tau'} + \frac{\vec q^2}{m_N^2}c_9^\tau c_{9}^{\tau'}
$$
$$
+\frac{\vec v_T^{\bot2}}{2} (c_{12}^\tau - \frac{q^2}{m_N^2} c_{15}^\tau) (c_{12}^{\tau'} - \frac{q^2 }{ m_N^2 } c_{15}^{\tau'})
\frac{\vec q^2}{2m_N^2 } \vec v_T^{\bot2} c_{14}^\tau c_{14}^{\tau'}]
+ v_T^{\bot2}c_{12}^\tau c_{12}^{\tau'}
\frac{\vec q^2}{m_N^2}c_{13}^\tau c_{13}^{\tau'}
]
$$
$$
R_{\Delta } ^{\tau \tau'} (\frac{\vec q^2}{m_N^2}  \vec v_T^{\bot2}, \frac{\vec q^2}{m_N^2})
=\frac{j_\chi (j_\chi+1)}{3} [c_5^\tau c_5^{\tau'} - c_8^\tau c_8^{\tau'} ]
$$
$$
R_{\Delta \Sigma'} ^{\tau \tau'} ( \vec v_T^{\bot2}, \frac{\vec q^2}{m_N^2})
=\frac{j_\chi (j_\chi+1)}{3} [c_5^\tau c_4^{\tau'} - c_8^\tau c_9^{\tau'} ]
$$

The eight WIMP responses above were derived from the six WIMP tensors of the EFT input by squaring matrix element and re-grouping terms. Notice that there are interference terms for $\Delta \Sigma'$ and $\Phi'' M...$ responses. It can be showed that other  interference terms vanish at LO.

On the other hand, the corresponding eight nuclear responses $W_M^{\tau\tau'}$ ...  were derived from the six nuclear operators $M_{JM;\tau}(q)...$ also by squaring the matrix element and re-grouping terms.  The nuclear operators  $M_{JM;\tau}(q)... $ are in turn the sum over all nucleons in nucleus of individual nucleonic operators. The individual nucleonic operators, however, were the results of multipole expansion discussed above, from the underlying nuclear modelling. 

Watching how one of the operators and coefficients journeyed through their way to the final expression might demonstrate the procedure. For example, the spin-dependent one, $c_4$ and $O_4$, were first re-grouped into the $\vec l_5$ WIMP axial vector density since all the operators in $\vec l_5$ are dotted with $S_N$ so all of these operators and coefficients enter Eq.() through  $\vec l_5$ in this way. Then, carried among other operators, $c_4O_4$ were carried from Eq.() to Eq.() when all nucleonic quantities were written explicitly in terms of operators.  The Hamiltonian density expression in Eq. () were then summed over nucleons and isospins. After spin sums, $c_4$ is hidden inside the expression of $R_\Sigma''$ as the square term and in $R_{\Delta\Sigma'}$ as the interference term of $O_4$ with $O_5$. Some terms which do not obey the assumed good parity and CP were killed at this stage. The other part of the WIMP-nucleus interaction is the nuclear responses. The spin of individual nucleons enter the final result through nuclear responses $\Sigma''_{JM;\tau}(q)$, and have to be summed over all nucleons. The use of Bessel spherical harmonics and Bessel vector spherical harmonics in the plane wave expansion then resulted in the expression for matrix element. 


Now, the traditional spin independent and spin dependent form factors are expressed in terms of the new response functions (nuclear matrix element) by
$$
|M_{SI}(0) |^2= \frac{4\pi}{2j_N+1} | \bra{j_N} | M_0(0) | \ket{j_N}| ^2
$$
$$
|M_{SD}(0)|^2 = \frac{4\pi}{2j_N+1}( | \bra{j_N} | \Sigma_1''(0) | \ket{j_N}| ^2+ | \bra{j_N} | \Sigma_1'(0) | \ket{j_N}| ^2)
$$

To summarise this section, in total, there are six response functions for nucleus in WIMP-nucleus elastic scattering, instead of two. The square of them generated six square matrix element plus two interference terms, while all others are killed by the good P and CP nuclear ground state assumption. 


Some new response functions are generated from the internal velocities of nucleons. They arise due to the composite nature of the nucleus, beyond the point limit. 
This includes the separation of longitudinal and transverse components of the spin response. These two contributions tends to be proportional to each other in the long wavelength limit. This is the traditional spin dependent operator. Therefore, there are now actually two separate responses, instead of one for the traditional SD. 

In the $ q^2 \rightarrow 0$ limit, all contributions from the EFT constants are zeroed except $c_1$ and $c_4$. Away from this limit, they start to appear. This demonstrate an important point of this section: if one uses expansion of $\vec q \cdot \vec x$ in the last section, they must be using the whole set of new operators for the expansion to be consistent. Otherwise, the expansion of $q$ is used to modify the form factors while the expansion of $q$ not used to add new operators. This demonstrate the reason why when working together with the last section, new sets of nuclear responses keeps the expansion consistent. 





\end{document}
